\documentclass{article}
\renewcommand*\familydefault{\sfdefault}
\usepackage[utf8]{inputenc}
\usepackage[T1]{fontenc}
\setlength{\textwidth}{481pt}
\setlength{\textheight}{650pt}
\setlength{\headsep}{10pt}
\usepackage{amsfonts}
\usepackage[T1]{fontenc}
\usepackage{palatino}
\usepackage{calrsfs}
\usepackage{geometry}
\geometry{ left=3cm, top=2cm, bottom=2cm, right=2cm}
\usepackage{xcolor}
\usepackage{amsmath}
\usepackage{tikz,tkz-tab}
\usepackage{cancel}
\usepackage{pgfplots}
\usepackage{pstricks-add}
\usepackage{pst-eucl}
\usepackage{amssymb}
\usepackage{icomma}
\usepackage{listings}
\begin{document}
\title{Informatique : DM sur la dynamique gravitationnelle}
\date{}
\author{Gatt Guillaume}
\maketitle
\renewcommand{\thesection}{\Roman{section}}
	\setlength{\parindent}{1.5cm}
\section{Quelques fonctions utilitaires }
{\bf 1.} a) $[1,2,3] + [4,5,6]=[1,2,3,4,5,6]$ \\
b) $2 *[1,2,3]=[1,2,3,1,2,3]$
\lstinputlisting[language=Python, firstline=1, lastline=20]{Dm3.py}
\section{Etude de schémas numériques}
\subsection{Mise en forme du problème }
{\bf 1.} On peut écrire l'équation (1) sous la forme : \\
$\forall t \in I, z'(t)=f(y(t))$ car $\forall t \in I, z(t)=y'(t)$ \\
{\bf 2.} On a : \\
$y(t_{i+1})= y(t_i) + (y(t_{i+1}) - y(t_i))$ \\
Or : $(y(t_{i+1}) - y(t_i))=\int_{t_i}^{t_{i+1}}y'(t)dt$ \\
$y(t_{i+1})=y(t_i)+ \int_{t_i}^{t_{i+1}}z(t)dt$ car $\forall t \in I, z(t)=y'(t)$\\
de même pour z(t) : \\
$z(t_{i+1})= z(t_i) + (z(t_{i+1}) - z(t_i))$ \\
Or : $(z(t_{i+1}) - z(t_i))=\int_{t_i}^{t_{i+1}}z'(t)dt$ \\
$z(t_{i+1})=z(t_i)+ \int_{t_i}^{t_{i+1}}f(y(t))dt$ car $\forall t \in I, z'(t)=f(y(t))$
\subsection{Schéma d'Euler explicite}
{\bf 1.} On approche les intégrales par la méthode des rectangles à gauche : \\
$y_{ i+1}=y_i + h z_i = y_0 + h\sum_{k=0}^{i}z_k $ \\
$z_{i+1}=z_i + h f(y_i)= z_0 + h\sum_{k=0}^{i} f(y_k) $ \\
\lstinputlisting[language=Python, firstline=22, lastline=37]{Dm3.py}
La fonction prend en paramètre f, la fonction donné par l'équation différentielle afin de calculer les $(z_i)$, $t_{min}, t_{max}$ et $n$, pour obtenir la liste des $(t_i)$ afin de pouvoir calculer de manière approché les intégrales, ainsi que $z_0$ et $y_0$ afin de pouvoir initialiser la récurrence. \\
{\bf 3.} a) On a l'équation : \\
$y''+ \omega^2y(t)=0$ \\
$y''(t) \times y'(t) + y'(t) \times \omega^2 y(t)=0$ \\
On reconnait $g'(x)=-f(x)=\omega^2x $ : \\
$y''(t) \times y'(t) + y'(t) \times g'(y(t))=0$ \\
On primitive l'équation avec une constante d'intégration $-E$ : \\
$\frac{1}{2} (y'(t))^2 + g(y(t))-E=0$ \\
$\frac{1}{2} (y'(t))^2 + g(y(t))=E$ \\
b) On a en remplaçant $y(t)$ par $y_i$ et $y'(t)$ par $z_i$ : \\
$E_i=\frac{1}{2} (z_i)^2 + g(y_i)$ \\
On a donc : \\
$E_{i+1}-E_i =(\frac{1}{2} (z_{i+1})^2 + g(y_{i+1}))-(\frac{1}{2} (z_i)^2 + g(y_i))$ \\
$E_{i+1}-E_i =\frac{1}{2} ((z_{i+1})^2-(z_i)^{\bf 2.} + g(y_{i+1})-g(y_i)$ \\
Comme $g'=-f$, $g(x)=\frac{1}{2} \omega^2 x^2$  on a : \\
$E_{i+1}-E_i = h^2  \omega^2 (\frac{\omega^2 y_i^2}{2}+\frac{z_i^2}{2})$ \\
$E_{i+1}-E_i =h^2  \omega^2 (\frac{1}{2} (z_i)^2 + g(y_i))$ \\
$E_{i+1}-E_i =h^2  \omega^2 E_i$ \\
c) On aurait : \\
$\forall i, \frac{1}{2} (z_i)^2 + g(y_i)=E$ \\
d) On  a si conservation de E : \\
$2E= (z_{i})^2 + \omega^2 {y_i}^2$ donc $E\geq 0$ \\
Si $E \neq 0$ :\\
$(\frac{(z_{i})}{\sqrt{2E}})^2 + (\frac{\omega y_i}{\sqrt{2E}})^2=1$ \\
On a une ellipse \\
Si $E=0$ : \\
$z_{i}^2 = - \omega^2 {y_i}^2$ Or $(z_i)^2 \geq 0$ et $\omega^2 {y_i}^2 \geq 0$ donc $z_i=0$ et $y_i=0$ on obtient un point de coordonnées $(0;0)$ \\
e) Si le schéma numérique satisfaisait la conservation de E on aurait une ellipse ou un point or on a une spirale. Comme les $y_i$ et $z_i$ ont une valeur plus grande que leurs valeurs réelles $y(t_i)$ et $z(t_i)$ l'écart entre la solution et le calcul approché s'agrandit à chaque calcul ce qui donne une spirale qui s'éloigne du centre. \\
\subsection{Schéma de Verlet}
\lstinputlisting[language=Python, firstline=38, lastline=53]{Dm3.py}
{\bf 2.} a) On a : \\
$E_{i+1}-E_i =\frac{1}{2} ((z_{i+1} -z_i) (z_{i+1} +z_i)) + \frac{\omega^2}{2}((y_{i+1}-y_i)(y_{i+1}+y_i)) $ \\
$E_{i+1}-E_i =\frac{h}{4}(f_i + f_{i+1})(2z_i +\frac{h}{2}(f_i+f_{i+1})) + \frac{\omega^2}{2}((hz_i + \frac{h^2}{2}f_i)(2y_i+hz_i + \frac{h^2}{2}f_i)$ \\
$E_{i+1}-E_i =\frac{h}{4}(f_i + f_{i+1})(2z_i +\frac{h}{2}(f_i+f_{i+1})) + \frac{\omega^2}{2}((hz_i + \frac{h^2}{2}f_i)(2y_i+hz_i + \frac{h^2}{2}f_i)$ \\
$E_{i+1}-E_i =\frac{h}{4}(f_i + f_{i+1})(2z_i +\frac{h}{2}(f_i+f_{i+1})) + \frac{\omega^2}{2}(h^2z_i^2+f_ih^2y_i+2hy_iz_i +O(h^3))$ \\
Or $f_{i+1}=-\omega^2(y_i+hz_i+\frac{h^2}{2}f_i)$ \\
$E_{i+1}-E_i =\frac{h}{4}(f_i + -\omega^2(y_i+hz_i+\frac{h^2}{2}f_i))(2z_i +\frac{h}{2}(f_i-\omega^2(y_i+hz_i+\frac{h^2}{2}f_i))) + \frac{\omega^2}{2}(h^2z_i^2+f_ih^2y_i+2hy_iz_i +O(h^3))$ \\
$E_{i+1}-E_i =\frac{1}{4}(hf_i + -\omega^2(hy_i+h^2z_i+ O(h^3))(2z_i +\frac{1}{2}(hf_i+-\omega^2(hy_i+h^ 2z_i+ O(h^3)))) + \frac{\omega^2}{2}(h^2z_i^2+f_ih^2y_i+2hy_iz_i +O(h^3))$ \\
$E_{i+1}-E_i =\frac{1}{8}(hf_i + -\omega^2(hy_i+h^2z_i+ O(h^3))(hf_i+-\omega^2(hy_i+h^ 2z_i+ O(h^3))) + \frac{\omega^2}{2}(f_ih^2y_i+O(h^3))$ car $f(y_i)=-\omega^2 y_i$ \\
$E_{i+1}-E_i = \frac{f_i}{8}(h^2f_i+-\omega^2(h^2y_i+O(h^3))) + \frac{\omega^2}{2}(\frac{2f_ih^2y_i}{4}+O(h^3))$ \\
$E_{i+1}-E_i=O(h^3)$ car $f(y_i)=-\omega^2y_i$ \\
b) L'allure du graphique est une ellipse ce qui est en accord avec une conservation de E. \\
c) Le schéma de Verlet conserve E
\section{Problème à N corps}
\subsection{Position du problème}
{\bf 1.}$\vec F_j= \sum_{k \neq j} \vec F_{k/j}$
\lstinputlisting[language=Python, firstline=55, lastline=71]{Dm3.py}
\subsection{Approche numérique}
{\bf 1.} {\bf position[i]} donne la position des objets au temps $t_i$, {\bf vitesse[i]} donne la vitesse des objets au temps $t_i$ \\
{\bf 2.} On a par le principe fondamentale de la dynamique appliqué à un corps j avec O l'origine du repère : \\
$ m_j (\overrightarrow{OP_j})'' = \vec F_j $ \\
$(\overrightarrow{OP_j})'' =\frac{ \vec F_j}{m_j} $ \\
On a donc comme l'équation (1) avec $f=\frac{ \vec F_j}{m_j}$ \\
\lstinputlisting[language=Python, firstline=73, lastline=124]{Dm3.py}
{\bf 4.}a) $ln(\tau_N)=Kln(N)$ avec $K=2$ \\
b) on a donc une complexité en $O(N^{\bf 2.}$ \\
{\bf 5.} a) etat suiv a une complexité en $O(N^3)$ \\
b) La complexité est plus grande que celle obtenue en III.2.4 \\
\lstinputlisting[language=Python, firstline=126]{Dm3.py}

\end{document}
